\section{Orthogonal Objects and Subquotients of the Object of Realizers}\label{sec:orth}

The first small complete category we will be examining is the one
determined by partial equivalences relations on the object of
realizers (in the $\eff$, the natural number object). Our methodology
for establishing completeness will be the following.
\begin{itemize}
\item First we will concretely define the internal category we're
  studying
\item Second, we will prove it to be weakly equivalent when
  externalized to the full subfibration of $\cod$ given by arrows
  orthogonal to $\Omega$
\item We shall show then that this subfibration is strongly complete.
\end{itemize}
All of these stepped when chained together are sufficient to give us
the weak completeness of our internal category. The key insight here
is the final step. We can, after all, define the internal category
that we're working with in $\set$. We can certainly define the
complete subfibration of arrows orthogonal to $\Omega$. In $\set$,
however, these fibrations will not be even weakly
equivalent. Unsurprisingly then, a great deal of our reasoning may be
carried out in any topos and it is only this final step which should
rely on the particulars of a realizability topos.

To begin with, the internal category that we'll be working with has as
the object of objects,
\[
  Q_0 = \{R \subseteq \pow{N} \mid (\forall X \in F.\ \exists x.\ x \in F)
  \mathrel{\wedge}
  (\forall X, Y \in F.\ (\exists x.\ x \in X \mathrel{\wedge} x \in Y) \implies X = Y)\}
\]
That is, the object of quotients of subsets of $N$. We can
equivalently define this as
\[
  Q_0 = \{R : \Omega^{N \times N} \mid ``R \text{ is transitive and symmetric.}"\}
\]
The isomorphism of course is the functional relation sending $F$ to
the PER where an element is identified with its equivalence
class. Next, consider the restriction of
$\in : X \times \pow{X} \to \Omega$ to $Q_0$. This situation is given
by the pullback diagram
\[
  \begin{tikzcd}
    Q \ar[d, "u"] \ar[r] & {\{(x, y) \mid x \in y\}} \ar[d, "\unchar{\in}"]\\
     \pow{\pow{N}} \times Q_0 \ar[r, rightarrowtail] & \pow{\pow{N}} \times \pow{\pow{\pow{N}}}
  \end{tikzcd}
\]
Then we have a map $Q \to Q_0$ given by $\pi_2 \circ u$. Let us then
consider $\full(\pi_2 \circ u)$ (recalling
Example~\ref{ex:internal:full}). To grasp this category intuitively,
remember that a global object is simply an equivalence relation on a
subset of $N$ and a global morphism, $f : S \to T$ maps equivalence
classes of $S$ to equivalences classes of $T$. Thus we have
successfully internalized subquotients of $N$. Our goal will be show
that this internal category is weakly complete. In order to do this,
we will switch gears entirely and begin to work with orthogonal
morphisms.
\begin{defn}\label{defn:orth:orth}
  We say that a morphism $f : A \to B$ is orthogonal to an object $C$
  if every commuting square has a unique diagonal\footnote{The
    observant reader will note that this is simply the statement that
    $!_{I^*(C)}$ is orthogonal to $!_{f}$ in $\eff/I$}
  \[
    \begin{tikzcd}
      B \times C \ar[d, swap, "\pi_1"] \ar[r] & A \ar[d, "f"]\\
      B \ar[r, equals] \ar[ur, dashed] & B
    \end{tikzcd}
  \]
\end{defn}
Let us define $\orth(A)$ to be the collection of indexed families
orthogonal to $A$. We first observe the following elementary fact
\begin{thm}\label{thm:orth:orthfibration}
  $\orth(A)$ is a complete subfibration of $\cod$
\end{thm}
\begin{proof}
  \todo{Essentially this means showing that products and equalizers
    preserve orthogonality, $\Pi$ restricts, and pullbacks preserve
    orthogonality}
\end{proof}
We will now spend a bit more time establishing properties related
purely to orthogonal objects so that we can characterize them more
concisely in our case.
\begin{lem}\label{lem:orth:orthsubobjects}
  If $A \epi 1$ and then $\orth(A)$ is closed under subobjects.
\end{lem}
\begin{proof}
  Suppose we have some $f : X \to I$ and $f' : X' \to I$ so that there
  is an $m : f' \mono f$. We wish to show that if $f$ is orthogonal to
  $A$ than so is $f'$. Let us localize our proof to $\eff/I$. It is
  now sufficient to show that $f'^{I^*(A)} \cong f'$ where we have
  assumed that $f^{I^*(A)} \cong f$. For this, we note that
  $f \to f^{I^*(A)}$ is a mono since $I^*(A) \epi 1$ holds. This gives
  us that the following is a pullback
  \[
    \begin{tikzcd}
      f' \ar[d, rightarrowtail] \ar[r, rightarrowtail] & {f'^{I^*(A)}} \ar[d, rightarrowtail, "m^{I^*(A)}"]\\
      f \ar[r, rightarrowtail, swap, "m'"] & {f^{I^*(A)}}
    \end{tikzcd}
  \]
  Now since the bottom arrow is an iso, so too is the top and we're
  done. In order to show this is a pullback we just have to show that
  if we have $x \in f$ and $\lambda y. z \in f'^{I^*(A)}$ so that
  $m^*(\lambda y. z) = \lambda y. x$ we then have that
  $\lambda y. m(z(y)) = \lambda y. x$ so we have that for all $y$,
  $m(z(y)) = x$ so this pair uniquely corresponds to a $z \in f'$ as
  required.
\end{proof}
We now turn to recharacterizing orthogonality in a slightly different
way that will allow us to prove a number of important closure
theorems.
\begin{lem}\label{lem:orth:orthpushout}
  In order to show that $f : X \to I$ is orthogonal to $A \epi 1$ if
  and only if that there is an $\iota$
  \[
    \begin{tikzcd}
      f^{I^*(A)} \times_I I^*(A) \ar[r, "\pi_1"] \ar[d, "ev"] & f^{I^*(A)} \ar[d, "\iota"]\\
      f \ar[r, equals] & f
    \end{tikzcd}
  \]
\end{lem}
\begin{proof}
  The only if direction is trivial because $f$ being orthogonal to $A$
  specifically implies that $\iota : f^{I^*(A)} \cong f$ must exist
  and such a map obviously forms a commutative square. For the if
  direction, if $\iota$ exists then it is clear that it is a right
  identity to the diagonal $f \to f^{I^*(A)}$ which is monotone since
  $A \epi 1$ giving us that $f \to f^{I^*(A)}$ is an iso as required.
\end{proof}
\begin{cor}\label{lem:orth:orthpushout2}
  The previous lemma can be stated as the following statement if the
  following two maps commute.
  \[
    \begin{tikzcd}
      {f^{I^*(A)} \times_I I^*(A)^2} \ar[r] &
      {f^{I^*(A)} \times_I I^*(A) \times_I I^*(A)} \ar[r, shift left=0.2em] \ar[r, shift right=0.2em] &
      {f^{I^*(A)} \times_I I^*(A)} \ar[r] &
      f
    \end{tikzcd}
  \]
\end{cor}
\begin{proof}
  \todo{Should be a simple elaboration of pushout conditions}
\end{proof}
In particular, we really wish for a theorem to characterize when
$\orth(A)$ is the same as $\orth(B)$ for some $A$ and $B$. This is
because the object $\Omega$ in $\eff$ is somewhat ungainly to work
with. It would be much more convenient if we could switch to studying
simpler objects (it will turn out that quotients of $\neg\neg$-sheaves
suffice).
\begin{lem}\label{lem:orth:orthequiv}
  If $B \epi A \epi 1$ so that $2 \to A$ inducing an epimorphism
  $B^A \epi B^2$ by composition then $\orth(A)$ is equal to
  $\orth(B)$.
\end{lem}
\begin{proof}
  First, we wish to show that if $f : X \to I$ is $B$-orthogonal than
  it is $A$-orthogonal. Suppose we have the following commutative diagram
  \[
    \begin{tikzcd}
      I^*(A) \ar[r, "g"] \ar[d] & f \ar[d]\\
      1 \ar[r, equals] & 1
    \end{tikzcd}
  \]
  We can rewrite this to
  \[
    \begin{tikzcd}
      I^*(B) \ar[d] \ar[r, twoheadrightarrow] &I^*(A) \ar[r, "g"] \ar[d] & f \ar[d]\\
      1 \ar[r, equals] \ar[urr, dashed, "g'"] &1 \ar[r, equals] & 1
    \end{tikzcd}
  \]
  And since $I^*(B) \epi I^*(A)$ we have that $g' \circ ! = g$ as well
  so we're done.

  Now instead suppose that $f : X \to I$ is $A$-orthogonal, we now
  wish to show that it is $B$-orthogonal. We note that
  \[
    \begin{tikzcd}
      {f^{I^*(B)} \times_I I^*(B)^{I^*(A)}} \ar[r] \ar[d, twoheadrightarrow] & f^{I^*(A)} \ar[d, "\iota^{-1}"]\\
      {f^{I^*(B)} \times_I I^*(B)^2} \ar[r, shift left=0.2em] \ar[r, shift right=0.2em] & f\\
    \end{tikzcd}
  \]
  Now we can conclude that the bottom two arrows are equal because the
  left arrow is epi and so applying
  Lemma~\ref{lem:orth:orthpushout2} gives us the desired conclusion.
\end{proof}
Having developed all of the supplementary theory, we are now in a
position to state and prove one of the main theorems we want about
subfibrations of orthogonal objects.
\begin{thm}\label{thm:orth:allequiv}
  The following fibrations are all equal to $\orth(\Delta 2)$
  \begin{enumerate}
  \item $\orth(\Delta S)$ where $S$ has at least two elements.
  \item $\orth(U)$ where $U$ is a uniform object with at least 2
    distinct sections
  \item $\orth(\Omega)$
  \end{enumerate}
\end{thm}
\begin{proof}
  \todo{Should prove this in full}.
\end{proof}
\begin{lem}\label{lem:orth:north}
  $N$ is orthogonal to $\Omega$.
\end{lem}
\begin{proof}
  \todo{This is necessary later}
\end{proof}

The remaining task is to prove that $\cat{Fam}(Q)$ is equivalent to
$\orth(\Omega) = \orth(\Delta 2)$. In order to start doing this, we
will begin with a more pleasant characterization of $\cat{Fam}(Q)$,
denoted in~\citet{Hyland:90} as a $(Q)$. The idea is that
$\cat{Fam}(Q) \simeq (Q) = \orth(\Delta 2)$, thus we break the
trickier task of establishing an equivalence into a more generic piece
of reasoning. The basic observation is that $\cat{Fam}(Q)$ is uniquely
determined by $v = \pi_2 \circ u : Q \to Q_0$, as any internal category of
the form $\full(f)$ is determined by $f$. It stands to reason then
that it ought to be possible to recover our internal category as the
full internal subcategory of $\cod : \eff^2 \to \eff$ given by all
arrows formed by pullback along $\pi_2 \circ u$.
\begin{thm}\label{thm:orth:altexternalization}
  $(Q)$ is a fibration with a cartesian inclusion into
  $\cod$.\footnote{This relies on some form of choice, namely a
    particular choice of pullbacks.}
\end{thm}
\begin{proof}
  In order to prove this, we will provide a full and faithful
  cartesian functor from $\cat{Fam}(Q)$ into $\cod$ with the image of
  $(Q)$. Let us send $f : I \to Q_0$ to the pullback $f^*(v)$. We then
  send a morphism
  \[
    \begin{tikzcd}
      I \ar[ddd, swap, "f"] \ar[rd, "h"] \ar[rr, "\alpha"] && J \ar[ddd, "g"]\\
      & Q_1 \ar[d, "\pi_2^*(v)^{\pi_1^*(v)}"] &\\
      & {Q_0 \times Q_0} \ar[dl, swap, "\pi_1"] \ar[dr, "\pi_2"]&\\
      Q_0 && Q_0
    \end{tikzcd}
  \]
  to the square
  \[
    \begin{tikzcd}
      f^*(Q) \ar[d, swap, "f^*(v)"] \ar[r, "h'"] & g^*(Q) \ar[d, "g^*(v)"]\\
      I \ar[r, swap, "\alpha"] & J
    \end{tikzcd}
  \]
  Where $h'$ is determined by the isomorphism
  \[
    \hom(\langle f, g \circ \alpha \rangle, \pi_2^*(v)^{\pi_1^*(v)}) \cong
    \hom(f^*(v), (g \circ \alpha)^*(v)) \cong
    \hom(\Sigma_\alpha f^*(v), g^*(v))
  \]
  But unfolding this, we see the left most hom set is precisely that
  contianing $h$s from the top diagram and the rightmost hom set is
  precisely that containing $h'$s from the bottom set. Thus we have an
  isomorphism between the two types of diagrams. This establishes that
  the diagram is full and faithful. Since by definition the image of
  this functor is $(Q)$ and it is obviously defined in a way so that
  it $\cod \circ (Q) = \family(Q)$\footnote{Here we are abusing
    notation and denoting the functor we are defining as $(Q)$}, it
  only remains to show that it is cartesian. This, however, follows
  automatically from the fact that the functor is full and faithful
  and appropriately fibered.
\end{proof}
Having established this characterization of $\cat{Fam}(Q)$ as
equivalent subfibration of $\cod$, we now work to show that this is
on-the-nose equal to $\orth(\Delta 2) = \orth(\Omega)$. In order to do
this, we will make use of the theory of $\neg\neg$ separated
objects. In particular, let us characterize a property which we shall
call $\dagger$.
\begin{defn}\label{defn:orth:coveredsep}
  We say that an object $X$ satisfies $\dagger$ if there is an
  $E \mono X \times N$ so that $E \epi X$ and $E \to N$ is a
  $\neg\neg$ sheaf in $\eff/N$.
\end{defn}
We now begin characterizing under what conditions $\dagger$ is
satisfied.
\begin{lem}\label{lem:orth:sepimpliesdagger}
  If $X$ is separated then $X$ satisfies $\dagger$.
\end{lem}
\begin{proof}
  Suppose that $X$ is separated, in the effective topos this means
  that it may be presented as $(X, \card{-})$ where
  \[
    x = y \in X \triangleq
    \begin{cases}
      \card{x} & x = y\\
      \emptyset & \text{otherwise}
    \end{cases}
  \]
  Let us now define $E$ as with the underlying set of $X \times N$.
  Since we want $E$ to be separated, it suffices to define
  $\card{-} : X \times N \to N$. Let us define this with
  \[
    \card{(x, n)} =
    \begin{cases}
      \{n\} & n \in \card{x}\\
      \emptyset & \text{otherwise}
    \end{cases}
  \]
  Since $X$ and $N$ are both separated, this relation is trivially
  functional with respect to their PERs so this gives rise to a
  subobject $E \mono X \times N$. Next we need to define an
  epimorphism $E \epi X$. We take the obvious functional relation
  defined for the map $(x, n) \mapsto x$. We can trivially realize
  this with $\lambda x. x$ so the properties of the resulting
  functional relation may be realized as well. It is trivial to show
  that this is epi because for all $x \in X$,
  $\card{x} \neq \emptyset$. Finally, we need to show that
  $f = (x, n) \mapsto n : E \to N$ is a $\neg\neg$-sheaf in $\eff/I$. It
  suffices to show that $f \mono N^*(\Delta X)$ is a closed subobject
  since $N^*$ is logical and thus preserves $\neg\neg$-sheaves. It
  suffices to show that
  \[
    E \mono X \times N \mono \Delta X \times N
  \]
  is closed but this is a trivial exercise in realizing the
  implication $\forall n, x.\ \neg\neg E\card{(x, n)} \to E\card{(x, n)}$;
  it is easily done with a projection.
\end{proof}
\begin{lem}\label{defn:orth:sepsliceimpliesdagger}
  If $X \to I$ is separated and $I$ is separated then $X \to I$
  satisfies $\dagger$ in $\eff/I$.
\end{lem}
\begin{proof}
  For this, we first note that an object is separated if and only if
  the unit $A \to a(A)$ of the sheafification adjunction is a
  monomorphism. Next, we can define $a$ in $\eff/I$ (written as $a_I$
  for disambiguation) in terms of $a$ in $\eff$ by the following
  pullback
  \[
    \begin{tikzcd}
      A' \ar[r] \ar[d, swap, "a_I(f)"] & a(A) \ar[d, "a(f)"]\\
      I \ar[r, swap, "\eta_I"] & a(I)
    \end{tikzcd}
  \]
  Now if $I$ is separated and $f : X \to I$ is separated then $X$ is
  separated by simple composition of these two facts. To see this,
  first note that the top side is the composition of two monomorphisms
  in the following diagram
  \[
    \begin{tikzcd}
      A \ar[d, swap, "f"] \ar[r, "\eta_f"] & A' \ar[r] \ar[d, swap, "a_I(f)"] & a(A) \ar[d, "a(f)"]\\
      I \ar[r, equals] & I \ar[r, swap, "\eta_I"] & a(I)
    \end{tikzcd}
  \]
  and since
  \[
    \begin{tikzcd}
      A \ar[rd, swap, "\eta_f"] \ar[rr, "\eta_A"]&& a(A) \ar[dd, dashed]\\
      &A' \ar[dr]&\\
      && a(A)
    \end{tikzcd}
  \]
  commutes using the universal property of $\eta_A$ we get that
  $\eta_A$ is a mono by the properties of a the sheaf and the fact
  that if $f \circ g$ is mono so is $g$.

  In any case, this is all to establish that $X$ is
  separated. Therefore, by Lemma~\ref{lem:orth:sepimpliesdagger} we
  have that there is an $E$ so that $E \mono X \times N$ and
  $E \epi N$ with $E \to N$ being a $\neg\neg$-sheaf. Next, we note
  that $f \times_I N^*(I) \cong X \times I$. Therefore, we have a mono
  \[
    \begin{tikzcd}
      E \ar[r, rightarrowtail] \ar[dr] &
      X \times N \ar[d, swap, "f \circ \pi_2"] \ar[r, equals] &
      X \times_I I^*(N) \ar[dl] \\
      &I&
  \end{tikzcd}
  \]
  Therefore we have a surjection $E \epi f$ given simply by
  composition with $\pi_1$. All that is left to show is that the
  following is a $\neg\neg$-sheaf.
  \[
    \begin{tikzcd}
      E \ar[r, rightarrowtail] \ar[dr] &
      X \times_I I^*(N) \ar[r, "\pi_2"] &
      I^*(N) \ar[dl] \\
      &I&
  \end{tikzcd}
  \]
  In order to show this it suffices to show that $f : E \to I^*(N)$ is
  a closed subobject of a sheaf, $a_I X \times N$, which is a sheaf
  since it is equal to $N^*(a_I X)$ and $N^*$ is logical. Let us first
  observe that we then have the pullback
  \[
    \begin{tikzcd}
      a_I X \times N \ar[r] \ar[d] & a(X \times N) \ar[d] \\
      I \times N \ar[r, swap, "\eta"] & a(I \times N)
    \end{tikzcd}
  \]
  Now we can extend this
  \[
    \begin{tikzcd}
      E \ar[d, swap, "f"] \ar[r, rightarrowtail] &
      X \times N \ar[d] \ar[r, rightarrowtail, "\eta"] &
      a_I(X) \times N \ar[r] \ar[d] &
      a(X) \times N \ar[d] \\
      I \times N \ar[r, equals] &
      I \times N \ar[r, equals] &
      I \times N \ar[r, swap, rightarrowtail, "\eta"] &
      a(I) \times N
    \end{tikzcd}
  \]
  We note that $E \mono a(X) \times N$ is closed because
  $a_I(X) \times N \mono a(X) \times N$ is closed. This in turn
  follows from the fact that $(f, g)$ is closed if $f$ and $g$ are
  both closed, obviously $1 : N \to N$ is closed and $a_I(X) \to a(X)$
  is closed because this is a morphism in the subcategory of
  $\neg\neg$-sheaves in $\eff/I$. Since
  $a_I(X) \times N \to a(X) \times N$ is mono, we know that
  \[
    \begin{tikzcd}
      a_I(X) \times N \ar[r, equals] \ar[d, equals] & a_I(X) \times N \ar[d, rightarrowtail] \\
      a_I(X) \times N \ar[r, rightarrowtail] & a(X) \times N
    \end{tikzcd}
  \]
  is a pullback and so $E \mono a_I(X) \times N$ is a closed subobject
  because closed subobjects are preserved by pullback.
\end{proof}
\begin{thm}\label{thm:orth:seporth}
  If $f : X \to I$ is orthogonal to $\Delta 2$ and is covered by
  $g : Y \to I$ with property $\dagger$, then $X \to I$ is a
  subquotient of $I^*(N)$.
\end{thm}
\begin{proof}
  Suppose that $h : E \to I$ is the object so that
  \[
    \begin{tikzcd}
      E \ar[rr, rightarrowtail] \ar[dr] & & Y \times_I I^*(N) \ar[dl]\\
      & I &
    \end{tikzcd}
  \]
  We can rewrite this diagram as
  \[
    \begin{tikzcd}
      E \ar[rr, rightarrowtail] \ar[ddr] & & Y \times_I I^*(N) \ar[d, twoheadrightarrow]\\
      & & X \times_I I^*(N) \ar[dl]\\
      & I &
    \end{tikzcd}
  \]
  We know that this map from $E \to X \times I^*(N)$ has a canonical
  factorization as a epi followed by a mono so we can complete this
  square
  \[
    \begin{tikzcd}
      E \ar[rr, rightarrowtail] \ar[d, twoheadrightarrow] & & Y \times_I I^*(N) \ar[d, twoheadrightarrow]\\
      E' \ar[rr, rightarrowtail] \ar[dr] & & X \times_I I^*(N) \ar[dl]\\
      & I &
    \end{tikzcd}
  \]
  Since $E \to Y \times_I I^*(N) \to Y$ is epi, as is
  $Y \times_I I^*(N) \to X \times_I I^*(N)$ we know that
  $E \to E' \to X \times_I I^*(N) \to X$ must also be epi. It is then a
  simple property of epimorphisms that $E' \to X \times_I I^*(N) \to X$
  is itself an epimorphism. Now, if we can show that
  $E' \to X \times_I I^*(N) \to I^*(N)$ is a monomomorphism, then we
  have
  \[
    \begin{tikzcd}
      E' \ar[rr, rightarrowtail] \ar[dr] & & X \times_I I^*(N) \ar[dl]\\
      & I &
    \end{tikzcd}
  \]
  $X \times_I I^*(N) \cong X \times N$ and $\orth(\Omega)$ is closed
  under finite limits by Theorem~\ref{thm:orth:orthfibration} we note
  that $X \times N$ is orthogonal to $\orth(\Omega)$. Here we have
  also used that $\orth(\Omega) = \orth(a(2))$ and that $N$ is
  orthogonal to $a(2)$ by Lemma~\ref{lem:orth:north}. Finally, this
  gives us that $E' \to I$ is orthogonal to $\Omega$ by
  Lemma~\ref{lem:orth:orthsubobjects}.
\end{proof}
\begin{thm}\label{defn:orth:seporth}
  The subfibrations $(Q)$ and $\orth(\Delta 2)$ are identical.
\end{thm}
\begin{proof}
  \todo{Really should prove}
\end{proof}
\begin{cor}
  $Q$ is an small, weakly complete category.
\end{cor}

%%% Local Variables:
%%% mode: latex
%%% TeX-master: "main"
%%% End:
