\section{Freyd's Theorem}\label{sec:freyd}

Freyd's theorem is a well-known result in category theory which states
that any small category which is complete for all small limits must be
a poset. This is the starting point for our development because the
nonconstructive argument that is used to prove this result suggests
that it might fail in a nonclassical setting.
\begin{thm}\label{thm:freyd:freyd}
  In any small category $\Ccat$ which is complete must be a poset.
\end{thm}
\begin{proof}
  Suppose we have such a $\Ccat$. We wish to show that for any
  $A, B \in \Ccat$ that $\card{\hom(A, B)} \le 1$. Let's suppose that
  this is not the case, namely, that we have a pair of objects $A$ and
  $B$ so that $\card{\hom(A, B)} \ge 2$. Let us distinguish two
  morphisms $a \neq b \in \hom(A, B)$ for later purposes.. Since
  $\Ccat$ is small we know that the collection of all morphisms is a
  set, let us say of cardinality $\kappa$. Let us then form the
  product $C = \prod_{i \in \kappa} B$. This must exist since we have
  assumed the existence of all small limits. Now consider
  $\hom(A, C)$. We know that $\hom(A, C) \subset X$ where $X$ is
  \[
    X = \{\langle x_i \rangle_{i \in \kappa}\mid \forall i.\ x_i \in \{a, b\}\}
  \]
  However, $\card{X} = 2^\kappa$, contradicting our assumption that
  the cardinality of the set of morphisms is $\kappa$.
\end{proof}

What is especially interesting is this result is that it can be
generalized from applying to just a boolean topos, such as $\set$ to
any topos which is based on top of $\set$ with an object of
generators! In particular, we can generalize this result to any
internal category to a Grothendieck topos. This result was concisely
proven by \citet{Gubkin:10} and is restated for the purposes of better
preservation.
\begin{thm}\label{thm:freyd:grothendieck}
  In any category $\Ccat$ internal to a topos $\Etop$ based on $\set$
  which is complete must be a poset.
\end{thm}
\begin{proof}
  This proof relies essentially on two properties of Grothendieck
  toposes, first, the existence of a global sections functor into set,
  $\amalg \dashv \Gamma$ and an object of generators $G$. Using these
  two properties, we can transfer any complete non-poset internal
  category, $C$, to $\set$. This is done in two steps, first we
  consider the category $C^G$, where the operations of the internal
  category are lifted to $C^G$ by the functor action of $-^G$.

  This internal category is then complete. In order to show this we
  must show that for any internal category $D$, the functor
  $\Delta : C^G \to (C^G)^D$ has a right ad joint. However, it is
  clear that $(C^G)^D \cong (C^D)^G$ so we may simply lift the adjoint
  we have, $C^D \to C$, from completeness of $C$. It is a simple
  matter to show that the appropriate equations for the adjoint
  diagram are preserved by the functor $-^G$.

  Next we apply the global sections functor to get an internal
  category in $\set$. It follows in fact that this category is
  internally complete as well because any we know that
  $\Delta : C^G \to (C^G)^{\amalg_S 1}$ has a right adjoint,
  $\lim_S : (C^G)^{\amalg_S 1} \to C^G$ giving us a morphism
  $\Gamma(\lim_S) : \Gamma((C^G)^{\amalg_S 1}) \to \Gamma(C^G)$. But
  of course, we have that
  \[
    \Gamma((C^G)^{\amalg_S 1}) \cong \hom(\amalg_S 1, C^G) \cong
    \hom(S, \Gamma(C^G))
  \]
  But $\hom(S, \Gamma(C^G)) \cong \Gamma(C^G)^S$ so we have the
  desired $\Gamma(\lim_S) : \Gamma(C^G)^S \to \Gamma(C^G)$ and it is
  routine to check that the appropriate equations are preserved.

  As a final step, we note that $\Gamma(C^G)$ a poset because
  $\Gamma(C^G)_1 \cong \hom(G, C_1)$ and if
  $\Gamma(\langle \partial_0, \partial_1 \rangle) : \hom(G, C_1) \to \hom(G, C_0 \times C_0)$
  was a monomorphism then
  \[
    \langle \partial_0, \partial_1 \rangle \circ f = \langle \partial_0, \partial_1 \rangle \circ g
  \]
  which $f, g : G \to C_1$ would imply $f = g$. However, we
  specifically assumed this was not the case by supposing that $C$ was
  not a poset, that is, by supposing precisely that
  $\langle \partial_0, \partial_1 \rangle$ was not a morphism and
  therefore that there was a $h, k : A \to C_1$ so that
  $\langle \partial_0, \partial_1 \rangle \circ h = \langle \partial_0, \partial_1 \rangle \circ k$
  so that $h \neq k$. Finally, since $h \neq k$ we must have a $g : G
  \to A$ so that $h \circ g \neq k \circ g$ but
  $\langle \partial_0, \partial_1 \rangle \circ h \circ g = \langle \partial_0, \partial_1 \rangle \circ k \circ g$
  holds, giving us the desired contradiction.
\end{proof}

%%% Local Variables:
%%% mode: latex
%%% TeX-master: "main"
%%% End:
