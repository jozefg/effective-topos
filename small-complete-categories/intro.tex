\section{Introduction}\label{sec:introduction}

In this note I'd like to discuss a remarkable discovery of the 1970s:
the existence of small, complete, non-poset internal categories in the
effective topos. This is a surprising discovery not because these
subcategories are particularly exotic; in fact they are quite
natural. What makes this surprising is that no Grothendieck topos
(importantly $\set$) can contain such an internal category. Therefore,
it is a direct result of the nonclassical metatheory that $\eff$
provides that we can construct such internal categories.

These categories are in fact interesting for reasons beyond their mere
existence. They provide a method for describing naive semantics for
type theories with impredicative polymorphism, notably System F, which
would otherwise be impossible to describe. In essence, the
completeness of the internal category means that $\forall$ may be
simply interpreted as the product over the object of objects,
something that's in fact contained in the internal category since it's
closed under such limits. A more detailed description of the semantics
of System F that this provides is a subject for a future note however.

In this note we will mainly be concerned with establishing the minimum
complete theory in order to describe the internal categories of
interest and prove their completeness. This is nontrivial since even
stating what completeness is requires a fair amount of delicate
consideration. In section~\ref{sec:freyd} we will start by proving
that this work is in fact worth it; these categories don't exist in
$\set$. In sections~\ref{sec:completeness} and~\ref{sec:completeness}
some of the underlying theory behind our constructions, including a
discussion on why the right notion of subcategory to consider is in
fact, an internal category. Finally, the main thrust of the note is
developed in sections~\ref{sec:orth} and~\ref{sec:modest} where we
prove the existence of two such internal categories, the category of
objects orthogonal two $\Omega$ and the category of so-called
\emph{modest sets}. It is sadly impossible for this note to be
entirely knowledge free, references to the appropriate topics in more
advanced category theory have been provided but it is assumed that the
reader is familiar with category theory as it is developed
in~\citet{MacLane:98}. Any assumptions behind this that are not
explicitly noted should be considered a defect and reported to the
author. The author is heavily indebted to Hyland, Robinson, and
Rosolini as the technical developments of this paper are more or less
a stripped down version of~\citet{Hyland:90}.
