\section{Internal Category Theory}\label{sec:internal}

Before continuing further, it would behoove us to take some time to
discuss and develop the theory of internal category theory to a
topos. Essentially, the idea is that \emph{normal} category theory can
be stated as simply category theory done internally to $\set$ which
opens the door to doing category theory internally to $\eff$. To begin
with, we will review the development of category theory internal to
any category with pullbacks and pay particular attention to the
question of internal limits.
\begin{defn}\label{defn:internal:internalcat}
  An internal category is a pair of objects $C_1$ and $C_0$ equipped
  with the following data.
  \begin{itemize}
  \item A pair of parallel maps $\partial_0, \partial_1 : C_1 \to C_0$
  \item A splitting for these maps $i : C_0 \to C_1$, explicitly,
    $\partial_0 \circ i = \partial_1 \circ i = 1$)
  \item Given the pullback
    \[
      \begin{tikzcd}
        C_2 \ar[r] \ar[d] & C_1 \ar[d, "\partial_1"]\\
        C_1 \ar[r, swap, "\partial_0"] & C_0
      \end{tikzcd}
    \]
    a map $m : C_2 \to C_1$.
  \end{itemize}
  Moreover, these will satisfy the following properties,
  \begin{itemize}
  \item $m \circ (m \times 1) = m \circ (1 \times m)$
  \item $\partial_0 \circ m = \partial_0 \circ \pi_1$
  \item $\partial_1 \circ m = \partial_1 \circ \pi_2$
  \item $m \circ \langle i \circ \partial_0, 1\rangle = 1$
  \item $m \circ \langle 1, i \circ \partial_1\rangle = 1$
  \end{itemize}
\end{defn}
This concept seems rather opaque at first glance, but it is in fact
just the categorization of the normal definition of category. For
instance:
\begin{example}
  Any small category is just a category internal to $\set$.
\end{example}
\begin{example}\label{ex:internal:full}
  Given a category $\Ccat$ which is has pullbacks and local
  exponentials and a morphism $f : A \to B$ in this category, the
  category $\full(a)$ is an internal category defined as follows.
  \begin{description}
  \item[Objects] The object of objects is simply $B$
  \item[Morphisms] The object of morphisms is given by the domain of
    $\pi_2^*(f)^{\pi_1^*(f)}$, an object in $\Ccat/B \times B$. Let us
    denote this object $B_1$. We then define
    $\partial_i : B_1 \to B \times B \overset{\pi_i}{\to} B$
  \end{description}
  We then must describe the identity and composition morphisms. In
  order to construct the identity morphism, it suffices to find a
  morphism from $\Delta : B \to B \times B$ to
  $\pi_2^*(a)^{\pi_1^*(a)}$ in $\Ccat/B \times B$. Doing this will
  immediately ensure that it is indeed a section for $\partial_0$ and
  $\partial_1$. By the universal property of an exponential, this is
  equivalent to finding a morphism in
  $\hom_{\Ccat/B \times B}(\Delta \times \pi_1^*(a), \pi_2^*(a))$.
  However, examination of $\Delta \times \pi_1^*(a)$ shows that it is
  equivalent to $\Sigma_\Delta a$ since
  $\Delta^* \circ \pi_1^* \cong 1$. Therefore, we need a morphism in
  \[
    \hom_{\Ccat/B \times B}(\Sigma_\Delta a, \pi_2^*(a)) \cong
    \hom_{\Ccat/B \times B}(a, \Delta^* \pi_2^*(a)) \cong
    \hom_{\Ccat/B \times B}(a, a)
  \]
  So we simply pick the identity morphism.

  Next we perform the same operation for composition. We need a
  morphism in $\hom_{\Ccat/B \times B}(b_2, \pi_2^*(a)^{\pi_1^*(a)})$
  where $b_2 : B_2 \to B \times B$ is defined as
  $\langle \partial_0 \circ p, \partial_1 \circ p' \rangle$
  with the domain given by the pullback
  \[
    \begin{tikzcd}
      B_2 \ar[r, "p'"] \ar[d, swap, "p"] & B_1 \ar[d, "\partial_0"]\\
      B_1 \ar[r, swap, "\partial_1"] & B
    \end{tikzcd}
  \]
  In order to derive composition, we note that
  \begin{align*}
    \hom(\langle f, g \rangle, \pi_2^*(a)^{\pi_1^*(a)})
    &\cong \hom(\Sigma_{\langle f, g \rangle} 1, \pi_2^*(a)^{\pi_1^*(a)})\\
    &\cong \hom(1, \langle f, g \rangle^*(\pi_2^*(a)^{\pi_1^*(a)}))\\
    &\cong \hom(1, (\langle f, g \rangle^*\pi_2^*)(a)^{\langle f, g \rangle^*\pi_1^*(a)})\\
    &\cong \hom(1, g^*(a)^{f^*(a)})\\
    &\cong \hom(f^*(a), g^*(a))
  \end{align*}
  Therefore, we merely need to find a morphism
  \[
    m \in \hom((\delta_0 \circ p)^*(a), (\delta_1 \circ p')^*(a))
  \]
  We note that by
  $1 \in \hom(\pi_2^*(a)^{\pi_1^*(a)}, \pi_2^*(a)^{\pi_1^*(a)})$ we
  have a morphism
  $f \in \hom(\partial_0^*(a), \partial_1^*(a))$ so we can obtain
  \[
    m \triangleq p^*(f) \circ p'^*(f)
  \]
  Which typechecks because we have that
  $\partial_1 \circ p = \partial_0 \circ p'$ by the definition of
  $B_2$.
\end{example}
It will be fruitful to develop a small amount of normal category
theory in this framework. For instance, we can define what it means to
be a poset
\begin{defn}\label{defn:internal:poset}
  An internal category is a poset if
  $\langle \partial_0, \partial_1 \rangle$ is a monomorphism.
\end{defn}
Intuitively, a category if is a poset if the function which strips
away all information from a morphism except its domain and codomain
does not lose any information. In other words, there is at most one
morphism of any given type. In order to define what it means for a
category to have limits and other universal constructions, it will be
simplest to establish a corresponding version of $\cat{Cat}$ for
internal categories: $\cat{Cat}(\Ccat)$. Then all of these
constructions can be defined by the existence of suitable
adjoints. Therefore, it is necessary to define corresponding internal
versions of a functor and natural transformation.
\begin{defn}\label{defn:internal:functor}
  An internal functor $F : C \to D$ is a pair of morphisms
  $F_0 : C_0 \to D_0$ and $F_1 : C_1 \to D_1$ so that $F_0$ and $F_1$
  commute with $i$, $m$, and $\partial_i$.
\end{defn}
\begin{defn}\label{defn:internal:naturaltrans}
  An internal natural transformation $\alpha : F \to G$ is a morphism
  $C_0 \to D_1$ so that the following diagrams commute
  \[
    \begin{tikzcd}
      &C_1 \ar[dl, swap, "{\langle \alpha \circ \partial_0, G_1 \rangle}"]
           \ar[dr, "{\langle F_1, \alpha \circ \partial_1 \rangle}"]&\\
      D_2 \ar[r, swap, "m"] & D_1 & D_2 \ar[l, "m"]
    \end{tikzcd}
    \begin{tikzcd}
      &C_0 \ar[d, "\alpha"] \ar[dl, swap, "F_0"] \ar[dr, "G_0"]&\\
      D_0 & D_1 \ar[l, "\partial_0"] \ar[r, swap, "\partial_1"] & D_0
    \end{tikzcd}
  \]
\end{defn}
The diagram on the right being that the arrows produced by the
natural transformation have the right type. The diagram on the left
is simply an expression of naturality.
\begin{thm}
  $\cat{Cat}(\Ccat)$ is a 2-category with internal categories,
  functors, and natural transformations.
\end{thm}
\begin{proof}
\end{proof}
Having fleshed out that this forms a 2-category, we know that it
automatically inherits the notion of adjoint functors and therefore,
equivalences that are describable in all 2-categories. We will
devote greater care to the question of equivalences in the next
section, but for now we will take the opportunity to make some obvious
definitions.
\begin{thm}\label{thm:internal:2categorical}
  The 2-category $\cat{cat}(\Ccat)$ has finite (co)limits and
  exponentials when the ambient category does.
\end{thm}
\begin{proof}
  For the sake of brevity, we will only prove the (more interesting)
  case for exponentials. Suppose that our ambient category has
  exponentials and limits, we then which to construct $C^D$ for two
  internal categories. The first of these, $(C^D)_0$ will be defined
  as
  \[
    \{\langle f_0, f_1 \rangle : C_0^{D_0} \times C_1^{D_1} \mid
    i = f_1 \circ i \mathrel{\wedge} f_0 \circ \partial_i = \partial_i \circ f_1
    \mathrel{\wedge} f_1 \circ m = m \circ \langle f_1, f_1 \rangle \}
  \]
  Here we are using the internal language of the category (which
  contains equality since $\Ccat$ is assumed to have limits). The
  object of morphisms is defined as
  \[
    \{(F, G, \alpha) : (C^D)_0 \times (C^D)_0 \times D_1^{C_0} \mid
    F_0 = \partial_0 \circ \alpha \mathrel{\wedge} G_0 = \partial_1 \circ \alpha
    \mathrel{\wedge}
    m \circ \langle \alpha \circ \partial_0, G_1 \rangle = m \circ \langle F_1, \alpha \circ \partial_1 \rangle\}
  \]
  We define
  \[
    \partial_i = \pi_i \qquad
    i = \langle 1, 1, \Lambda i \rangle
  \]
  multiplication is arduous to write out but is simply the standard
  composition of natural transformations. It is immediate by
  definition that $C^D$ satisfies the required properties for
  exponentials.
\end{proof}
As a final construction, we will concern ourselves with the conversion
of an internal category to a particular fibration on the
category. This will be essential for the remainder of our technical
development because it is much more convenient to prove the
completeness of fibrations rather than work with cumbersome diagrams
for internal categories. Furthermore, the fibration, or
\emph{externalization}, of an internal category shares many essential
properties with its internal category.
\begin{defn}\label{defn:internal:externalization}
  The \emph{externalization} of an internal category $C \in \Ccat$ is
  a functor $\family(C) : \cat{Fam}(C) \to \Ccat$ where
  \begin{description}
  \item[Objects] Objects of $\cat{Fam}(C)$ are maps $I \to C_0$ in
    $\Ccat$.
  \item[Morphisms] Morphisms of $\cat{Fam}(C)(f, g)$ are pairs of maps
    $(\alpha, h)$ where $\alpha : \dom(f) \to \dom(g)$ and a map $h :
    I \to C_1$ so the following commutes
    \[
      \begin{tikzcd}
        I \ar[dd, swap, "f"] \ar[rd, "h"] \ar[rr, "\alpha"] && J \ar[dd, "g"]\\
        & C_1 \ar[dl, swap, "\partial_0"] \ar[dr, "\partial_1"] &\\
        C_0 && C_0
      \end{tikzcd}
    \]
    In other words, it is an $I$-indexed collection of morphisms which
    map one family of $C$ objects onto the new family.
  \end{description}
\end{defn}

\begin{thm}\label{thm:internal:externalization}
  $\family(C)$ is a split fibration.
\end{thm}
\begin{proof}
  \todo{Should prove this in full}
\end{proof}

%%% Local Variables:
%%% mode: latex
%%% TeX-master: "main"
%%% End:
